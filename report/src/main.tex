%! Author = marcus
%! Date = 10.12.24

\documentclass[12pt,a4paper]{article}

\usepackage[backend=biber,style=numeric]{biblatex}
\usepackage[utf8]{inputenc}
\usepackage[english]{babel}
\usepackage{graphicx}
\usepackage{geometry}
\usepackage{hyperref}

\addbibresource{./references.bib}

\geometry{margin=2.5cm}

\begin{document}

  \begin{titlepage}
    \begin{center}
      \vspace*{2cm}
      {\huge\bfseries Movie Finder\par}
      \vspace{2cm}
      {\Large Group Number: 08\par}
      \vspace{1.5cm}
      {\large\bfseries Group Members:\par}
      \vspace{0.5cm}
      {\large
      Patrick Eckel\par
      Marcus Gugacs\par
      Martin Tobias Klug\par
      Lukas Leitner\par
      }
      \vfill
      {\large \today\par}
    \end{center}
  \end{titlepage}

  \tableofcontents
  \newpage


  \section{Introduction}

  This section provides an overview of the project, including the motivation, and the research question.

  \subsection{Motivation}

  The digital entertainment landscape has transformed, with unprecedented content production and availability.
  However, this abundance of choice has become a burden for consumers.
  Streaming platforms prioritize user retention
  over satisfaction, resulting in suboptimal viewing experiences.
  Content discovery is challenging due to cognitive load from navigating extensive catalogs across multiple services.
  Existing recommendation systems exhibit biases towards platform-specific content and popular titles, wasting user time.
  A user-centric approach is needed, prioritizing user time and preferences over platform metrics.

  \subsection{Research Question}

  This project addresses a central research question:
  How can we build a central system using a transformer-based architecture to process natural language queries,
  to generate personalized movie recommendations and efficetly reduce the users effort of finding matching content?

  \noindent This question encompasses advanced natural language processing, maintaining recommendation relevance, and system
  adaptability through user feedback. It directly addresses creating a more efficient and user-centric content discovery
  platform while acknowledging technical complexity in processing user preferences and ensuring system responsiveness.


  \section{Related Work}

  This section provides an overview of the literature and a brief discussion regarding our project.
  This section starts with a brief overview of the literature, followed by a discussion of our actual project implementation.

  \subsection{Literature}

  There are several approaches to movie recommendation systems.
  For the literature overview we focus on literature matching the context of the ``Advanced Information Retrieval'' course.
  One paper espcially caught our attention, as it uses a transformer-based architecture to generate movie recommendations,
  but also discusses foundtaional steps like data preprocessing and dimensionality reduction.
  This is especially of interest to us, as the whole topic is novel to us and we are looking for a good starting point.

  \noindent One team of researchers \cite{Iglesias-pardo-lopez-quintero-2024} proposes a movie recommendation system that uses a
  transformer-based embeddings space. To achieve this it was necessary to perform dimensionality reduction.
  The first thing, as in many data science projects, is to preprocess the data.
  In the case of the discussed project the dataset of 45.466 movies was reduced to just 11.236 \cite{Iglesias-pardo-lopez-quintero-2024}.
  The features of the movies where transformed to a matrix consisting of id, original title, release date,
  original language, runtime, vote average, vote count, popularity, poster path, production companies, genres, overview,
  keywords, cast, director and release year \cite{Iglesias-pardo-lopez-quintero-2024}.
  To generate meaningful embeddings of sentences the a variation of the BERT-family of models was used \cite{Iglesias-pardo-lopez-quintero-2024}.
  Through the usage of dimensionality reduction they were able to visualize the movies on a 2D-plane \cite{Iglesias-pardo-lopez-quintero-2024}.

  \noindent The mentioned steps already provide good insight into a starting point for such a system.
  This was a good orientation for our project, but the actual user-interaction is highly different than our approach.
  In the case of the mentioned project the user is able to select 10 movies and the system will then generate a recommendation \cite{Iglesias-pardo-lopez-quintero-2024}.
  Afterwards the user is asked to rate the recommendation and the system will generate a new recommendation based on the feedback \cite{Iglesias-pardo-lopez-quintero-2024}.

  \subsection{Project Discussion}

  After reviewing the literature, we browsed through \href{https://www.huggingface.com/}{HuggingFace} to find datasets,
  models and other resources to build our system.
  This was also the most informative part of our own research as we gathered lots of ideas and insights from the community
  and through the extensive collection of readmes and whitepapers on the platform.
  This sophisticated movie recommendation system uses multiple advanced methods and domain-specific knowledge
  to deliver personalized movie recommendations. It strikes a balance between computational efficiency, accuracy, and
  user experience.
  The system employs a weighted ensemble of five key parts:
  \begin{itemize}
    \item Semantic Analysis (40\%): Uses the MPNet-based SentenceTransformer model to capture deep contextual understanding of movie descriptions and user preferences.
    \item TF-IDF Similarity (40\%): Captures specific terminology and phrases.
    \item Emotion Analysis (15\%): Using text-classification as a foundation the emotional accordance with the movie content and the mood choice by the user is evaluated.
    \item Vote Average Metrics (2,5\%): One out of two factors to ensure recommendations remain somewhat connected to mainstream appeal.
    \item Popularity Metrics (2,5\%): One out of two factors to ensure recommendations remain somewhat connected to mainstream appeal.
  \end{itemize}

  \noindent The system’s robustness is enhanced through efficient text preprocessing, caching, batch processing and
  GPU (CUDA, MPS) acceleration.

  \noindent Recommendation Generation Process:
  \begin{itemize}
    \item Filtering dataset based on trivial criteria (e.g.,\ language, genres).
    \item Initial query processing combines user mood preferences with notes.
    \item Multi-stage similarity computation generates semantic embeddings for the query and movie corpus, then calculates TF-IDF similarity, sentiment alignment, and popularity normalization.
    \item Final ranking selects top recommendations based on weighted scores.
  \end{itemize}

  \noindent This system is well-suited for streaming platform recommendation engines, personalized content curation, movie discovery
  and educational purposes.
  Due to it's modular design (frontend / backend strictly separated) it can be easily integrated into existing systems.
  The frontend is optionally deployable as the backend can be integrated into any existing system due to interfacing
  using a easy-to-understand REST API.


  \section{Experiments and Results}

  To evaluate the performance, feasibility, and efficiency of our system, a series of experiments were conducted.
  A standardized questionnaire was employed to assess the system’s performance.
  Each group member independently evaluated the system.
  This evaluation process was iterated over three days to identify a movie for each evening’s viewing.
  The individual evaluation reports can be found in the appendix folder within the
  \href{https://github.com/IImpaq/air-2024/appendix}{project repository}.
  The evaluation results of each group member were averaged to provide a comprehensive overview of the system’s performance.
  Our finalized results and evaluation is presented in the following paragraph.

  \noindent TODO Add evaluation results and appendix folder later

  \section{Conclusion}

  In the project we managed to produce highly usable and efficient movie recommendations.
  Nonetheless, there is still a lot of tweaking and fine-tuning to be done.
  Minor changes can lead to significant improvements in the user experience as we already noticed during development.

  \printbibliography

\end{document}
