%! Author = marcus
%! Date = 10.12.24

\documentclass[12pt,a4paper]{article}

\usepackage[utf8]{inputenc}
\usepackage[english]{babel}
\usepackage{graphicx}
\usepackage{geometry}
\usepackage{hyperref}

\geometry{margin=2.5cm}

\begin{document}

  \begin{titlepage}
    \begin{center}
      \vspace*{2cm}
      {\huge\bfseries Movie Finder\par}
      \vspace{2cm}
      {\Large Group Number: 08\par}
      \vspace{1.5cm}
      {\large\bfseries Group Members:\par}
      \vspace{0.5cm}
      {\large
      Patrick Eckel\par
      Marcus Gugacs\par
      Martin Tobias Klug\par
      Lukas Leitner\par
      }
      \vfill
      {\large \today\par}
    \end{center}
  \end{titlepage}

  \tableofcontents
  \newpage


  \section{Introduction}

  This section provides an overview of the project, including the motivation, and the research question.

  \subsection{Motivation}

  The digital entertainment landscape has transformed, with unprecedented content production and availability.
  However, this abundance of choice has become a burden for consumers. Streaming platforms prioritize user retention
  over satisfaction, resulting in suboptimal viewing experiences.

  \noindent Content discovery is challenging due to cognitive load from navigating extensive catalogs across multiple services.
  Existing recommendation systems exhibit biases towards platform-specific content and popular titles, wasting user time.
  Industry data shows the average user spends 20 minutes per session browsing for content, accumulating lost time.

  \noindent Factors exacerbate the problem:
  \begin{itemize}
    \item Accelerating content production with billions invested in new content.
    \item Algorithmic recommendations prioritizing platform engagement over user satisfaction.
    \item Limited cross-platform content discovery solutions.
    \item Inadequate consideration of contextual viewing preferences.
    \item Lack of efficient recommendation systems.
  \end{itemize}

  \noindent A user-centric approach is needed, prioritizing user time and preferences over platform metrics.

  \subsection{Research Question}

  This project addresses a central research question:
  How can we build a central system using a transformer-based architecture to process natural language queries,
  to generate personalized movie recommendations and efficetly reduce the users effort of finding matching content?

  \noindent This question encompasses advanced natural language processing, maintaining recommendation relevance, and system
  adaptability through user feedback. It directly addresses creating a more efficient and user-centric content discovery
  platform while acknowledging technical complexity in processing user preferences and ensuring system responsiveness.


  \section{Related Work}

  This section provides an overview of the literature and a brief discussion regarding our project.

  \subsection{Literature}

  TODO Add literature here.

  \subsection{Project Discussion}

  This sophisticated movie recommendation system uses multiple advanced methods and domain-specific knowledge
  to deliver personalized movie recommendations. It strikes a balance between computational efficiency, accuracy, and
  user experience.
  The system employs a weighted ensemble of four key parts:
  \begin{itemize}
    \item Semantic Analysis (40\%): Uses the MPNet-based SentenceTransformer model to capture deep contextual understanding of movie descriptions and user preferences.
    \item TF-IDF Similarity (25\%): Captures specific terminology and phrases.
    \item Sentiment Analysis (25\%): Provides emotional resonance matching between user mood preferences and movie content.
    \item Popularity Metrics (10\%): Ensures recommendations remain somewhat connected to mainstream appeal.
  \end{itemize}

  \noindent The system’s robustness is enhanced through efficient text preprocessing, caching, batch processing and
  GPU (CUDA, MPS) acceleration.

  \noindent Recommendation Generation Process:
  \begin{itemize}
    \item Filtering dataset based on trivial criteria (e.g.,\ language, genres).
    \item Initial query processing combines user mood preferences with notes.
    \item Multi-stage similarity computation generates semantic embeddings for the query and movie corpus, then calculates TF-IDF similarity, sentiment alignment, and popularity normalization.
    \item Final ranking selects top recommendations based on weighted scores.
  \end{itemize}

  \noindent This system is well-suited for streaming platform recommendation engines, personalized content curation, movie discovery
  and educational purposes.
  Due to it's modular design (frontend / backend strictly separated) it can be easily integrated into existing systems.
  The frontend is optionally deployable as the backend can be integrated into any existing system due to interfacing
  using a easy-to-understand REST API.


  \section{Experiments and Results}

  To evaluate the performance, feasibility, and efficiency of our system, a series of experiments were conducted.
  A standardized questionnaire was employed to assess the system’s performance. Each group member independently evaluated
  the system.
  The evaluator initially considered the movies (genres, actors, mood) they would like to watch. They then completed the
  questionnaire to document their expectations for recommendations.
  Next, the evaluator navigated through the application to locate the recommended movies. Finally, they completed the
  questionnaire once more to document the actual recommendations received.
  A subjective comparison of the anticipated and actual recommendations was conducted. This comparison encompassed the
  following aspects:
  \begin{itemize}
    \item To what extent did the system comprehend the user’s query?
    \item Were there any unfamiliar movies recommended by the system?
    \item How accurately did the system recommend movies that aligned with the user’s query?
    \item To what extent did the system recommend movies that the user would find enjoyable?
    \item To what extent did the system recommend movies that the user was already familiar with?
  \end{itemize}

  \noindent This evaluation process was repeated over five days to identify a movie for each evening’s viewing.


  \section{Conclusion}

  TODO Add key takeaways and future work.

\end{document}
