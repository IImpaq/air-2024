%! Author = marcus
%! Date = 04.11.24

\documentclass[12pt,a4paper]{article}

\usepackage[utf8]{inputenc}
\usepackage[english]{babel}
\usepackage{graphicx}
\usepackage{geometry}
\usepackage{hyperref}

\geometry{margin=2.5cm}

\begin{document}

  \begin{titlepage}
    \begin{center}
      \vspace*{2cm}
      {\huge\bfseries Project Title\par}
      \vspace{2cm}
      {\Large Group Number: X\par}
      \vspace{1.5cm}
      {\large\bfseries Group Members:\par}
      \vspace{0.5cm}
      {\large
      Name 1 (Role)\par
      Name 2 (Role)\par
      Name 3 (Role)\par
      Name 3 (Role)\par
      }
      \vfill
      {\large \today\par}
    \end{center}
  \end{titlepage}

  \tableofcontents
  \newpage

  \section{Abstract}

  \section{Idea}

  Nowadays the amount of content to consume is enormous.
  It is hard for many to find movies they actually want to watch in a sea of irrelevant content.
  Many streaming providers and online distributors would have the capabilities to provide a concise selection to their users.
  Many of them choose not to.
  The reasons aren't clear, but maybe it is required to bind users to their platform as long as possible.
  The users are recommended a wave of mideocor content that is nowhere close to what they actually want to see at a given point in time.
  Our idea is to value the time of humans by making it easier for them to actually watch the movies they are interested in.
  To achieve that we want to build a platform where a user can explicitely describe and define their taste in movies, in general or just what they want to watch right now.
  The system is then working hard by working its way through a vast movie database to find an appropriate and concise selection.
  The top results are then displayed to the user with brief descriptions of the content of a movie, the genre, actors, release data and title.
  If a user is not happy with the provided content the system adapts and searches through the database again using the modified criteria.
  Otherwise, if the user found a movie they like, we try to minmize the hurdle to start a movie by directly recommending a few services which provide the selected content.
  The user can just hop onto their favorite platform with the click of a button and stream/buy the movie and enjoy it.

  \subsection{Goal}

  \section{Main Task}

  \section{Dataset and Processing}
  \subsection{Data Sources}

  \subsection{Data Processing}


  \section{Methods and Models}

  \subsection{Methodology}
  \subsection{Implementation}

  \section{Evaluation}
  \subsection{Metrics}

  \subsection{Results}

  \subsection{Discussion}


\end{document}
