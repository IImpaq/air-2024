%! Author = marcus
%! Date = 04.11.24

\documentclass[12pt,a4paper]{article}

\usepackage[utf8]{inputenc}
\usepackage[english]{babel}
\usepackage{graphicx}
\usepackage{geometry}
\usepackage{hyperref}

\geometry{margin=2.5cm}

\begin{document}

  \begin{titlepage}
    \begin{center}
      \vspace*{2cm}
      {\huge\bfseries Project Title\par}
      \vspace{2cm}
      {\Large Group Number: X\par}
      \vspace{1.5cm}
      {\large\bfseries Group Members:\par}
      \vspace{0.5cm}
      {\large
      Name 1 (Role)\par
      Name 2 (Role)\par
      Name 3 (Role)\par
      Name 3 (Role)\par
      }
      \vfill
      {\large \today\par}
    \end{center}
  \end{titlepage}

  \tableofcontents
  \newpage

  \section{Abstract}

  \section{Idea}

  Nowadays the amount of content to consume is enormous.
  It is hard for many to find movies they actually want to watch in a sea of irrelevant content.
  Many streaming providers and online distributors would have the capabilities to provide a concise selection to their users.
  Many of them choose not to.
  The reasons aren't clear, but maybe it is required to bind users to their platform as long as possible.
  The users are recommended a wave of mideocor content that is nowhere close to what they actually want to see at a given point in time.
  Our idea is to value the time of humans by making it easier for them to actually watch the movies they are interested in.
  To achieve that we want to build a platform where a user can explicitely describe and define their taste in movies, in general or just what they want to watch right now.
  The system is then working hard by working its way through a vast movie database to find an appropriate and concise selection.
  The top results are then displayed to the user with brief descriptions of the content of a movie, the genre, actors, release data and title.
  If a user is not happy with the provided content the system adapts and searches through the database again using the modified criteria.
  Otherwise, if the user found a movie they like, we try to minmize the hurdle to start a movie by directly recommending a few services which provide the selected content.
  The user can just hop onto their favorite platform with the click of a button and stream/buy the movie and enjoy it.

  \subsection{Goal}
  Our goal is to value the time of humans by just recommending them movies they actually want to watch without trying to force them to search through a catalogue of irrelevant content.
  The necessity for such a service will just increase over time as the amount of movies/series will start to grow.
  There are already more tv shows created in a given period of time than ever before and the trend seems to stay.

  \section{Main Task}

  The focus of our work is to develop an advanced information retreival system for movie recommendations using a transformer-based architecture.
  To achieve this, our core component will be a combination of transformer models:
  \begin{itemize}
    \item to perform text encoding
    \item for text summarization
    \item as a sentence transformer for semantic similarity
  \end{itemize}

  \noindent Technically, the retrieval pipeline will work approximately as follows:
  \begin{itemize}
    \item Processing the user query regarding their preferences.
    \item Document encoding of the movie preferences retreived from the dataset
    \item Computing the similarity of the encoded user query and the encoded movie document
    \item Ranking the results based on relevance to the user
  \end{itemize}

  \noindent After the first phase of the retrieval pipeline is finished the user can choose if they are satisfied with the result or they can select irrelevant recommendations.
  If a selection of irrelevant content is made, the system will adapt and provide a new selection based on the changing requirements.
  This should provide the highest quality recommendations with the least amount of effort from the user side.

  \section{Dataset and Processing}

  This section goes into detail regarding the sources of used data, as well as how such data is processed.

  \subsection{Data Sources}

  We are using the ``wykonos/movies'' dataset as our movie database.
  It is loaded through Hugging Face.
  The data set contains the following features:
  id, title, genre, original language, overview, popularity, production companies, release date, budget / revenue, runtime, status, tagline, vote (avg) \& vote (count), credits, keywords.

  We also need a provider for the captions/subtitles such that we can generate an informative summarization/recommendation based on the actual content of the movie.
  This provider will be contacted and their data fetched through the backend of the system.
  Such a feature should be easily implementable using a simple REST-API.
  We are, at the time of writing this document, still searching for a cheap/free option to fullfill this requirement. (TODO)

  \subsection{Data Processing}

  Before using the dataset we need to pre-process it.
  This is done through a data ``cleaning'' step.
  During this procedure, the missing values (NaN) are replaced with empty strings to avoid future issues.
  In the next step, each of the entries in the dataset is converted to strings.

  \section{Methods and Models}

  This section tries to explain a technical deep dive of how our previously defined IR problem will be solved.

  \subsection{System Architecture}

  TODO

  \subsection{Implementation}

  The whole backend of the system will be implemented using python.
  There are many useful libraries and tools which are helpful to work on the given problem, examples include but are not limited to: pytorch, pandas, numpy, scipy.
  The first step is to perform the required input processing (encoding).
  For that, an optimized transformer model will be used.
  Afterwards, using the model generated using the tokenized input, we use mean pooling of the last hidden states for similarity computation later on.
  Additionaly to the user input, also the the movie descriptions must be encoded.
  This then allows for computation of a similarity measurement between user input and movie description (e.g. cosine similarity).
  To provide the user only with the most relevant recommendations, a sophisticated ranking mechanism based on the similarity scores will be applied.
  Now only the top 3 results, enriched using a variety of additional movie metadata and a summary/recommendation of the content can be shown to the user.
  To provide the user with the summary we must first make a request to a caption/subtitle provider and then use the returned data in a summerization pipeline with an apprioriate model.
  The recommendation will then be visualized in the frontend which will be implemented using a modern web technology stack, including but not limited to: NextJS.
  The recommendation can then be evaluated by the user and in the case of dissatisfaction be re-evaluated by the model to update the recommendation using a selection of irrelevant movies by the user.

  \subsection{Optimization questions}
  Due to the size of the dataset a variety of optimization questions come to mind.
  This includes topics like memory management strategies, increased scalability using batch processing and hardware acceleration (CUDA/MPS)

  \section{Evaluation}
  \subsection{Metrics}

  \subsection{Results}

  \subsection{Discussion}


\end{document}
